\nonstopmode{}
\documentclass[a4paper]{book}
\usepackage[times,inconsolata,hyper]{Rd}
\usepackage{makeidx}
\usepackage[utf8]{inputenc} % @SET ENCODING@
% \usepackage{graphicx} % @USE GRAPHICX@
\makeindex{}
\begin{document}
\chapter*{}
\begin{center}
{\textbf{\huge Package `ggrastr'}}
\par\bigskip{\large \today}
\end{center}
\begin{description}
\raggedright{}
\inputencoding{utf8}
\item[Type]\AsIs{Package}
\item[Title]\AsIs{Raster Layers for 'ggplot2'}
\item[Version]\AsIs{0.1.9}
\item[Description]\AsIs{Provides a set of geoms to rasterize only specific layers of the plot while simultaneously keeping all labels and text in vector format. This allows users to keep plots within the reasonable size limit without loosing vector properties of the scale-sensitive information.}
\item[License]\AsIs{MIT + file LICENSE}
\item[Encoding]\AsIs{UTF-8}
\item[LazyData]\AsIs{true}
\item[Imports]\AsIs{ggplot2 (>= 2.1.0), Cairo (>= 1.5.9), ggbeeswarm}
\item[Depends]\AsIs{R (>= 3.2.2)}
\item[RoxygenNote]\AsIs{7.1.0}
\item[Suggests]\AsIs{rmarkdown, knitr}
\item[VignetteBuilder]\AsIs{knitr}
\item[URL]\AsIs{}\url{https://github.com/VPetukhov/ggrastr}\AsIs{}
\item[BugReports]\AsIs{}\url{https://github.com/VPetukhov/ggrastr/issues}\AsIs{}
\item[NeedsCompilation]\AsIs{no}
\item[Author]\AsIs{Viktor Petukhov [aut, cph], Evan Biederstedt [cre, aut]}
\item[Maintainer]\AsIs{Evan Biederstedt }\email{evan.biederstedt@gmail.com}\AsIs{}
\end{description}
\Rdcontents{\R{} topics documented:}
\inputencoding{utf8}
\HeaderA{geom\_beeswarm\_rast}{This geom is similar to \code{\LinkA{geom\_beeswarm}{geom.Rul.beeswarm}}, but creates a raster layer}{geom.Rul.beeswarm.Rul.rast}
%
\begin{Description}\relax
This geom is similar to \code{\LinkA{geom\_beeswarm}{geom.Rul.beeswarm}}, but creates a raster layer
\end{Description}
%
\begin{Usage}
\begin{verbatim}
geom_beeswarm_rast(
  mapping = NULL,
  data = NULL,
  stat = "identity",
  position = "quasirandom",
  priority = c("ascending", "descending", "density", "random", "none"),
  cex = 1,
  groupOnX = NULL,
  dodge.width = 0,
  ...,
  na.rm = FALSE,
  show.legend = NA,
  inherit.aes = TRUE,
  raster.width = NULL,
  raster.height = NULL,
  raster.dpi = 300
)
\end{verbatim}
\end{Usage}
%
\begin{Arguments}
\begin{ldescription}
\item[\code{mapping}] Set of aesthetic mappings created by \code{\LinkA{aes()}{aes()}} or
\code{\LinkA{aes\_()}{aes.Rul.()}}. If specified and \code{inherit.aes = TRUE} (the
default), it is combined with the default mapping at the top level of the
plot. You must supply \code{mapping} if there is no plot mapping.

\item[\code{data}] The data to be displayed in this layer. There are three
options:

If \code{NULL}, the default, the data is inherited from the plot
data as specified in the call to \code{\LinkA{ggplot()}{ggplot()}}.

A \code{data.frame}, or other object, will override the plot
data. All objects will be fortified to produce a data frame. See
\code{\LinkA{fortify()}{fortify()}} for which variables will be created.

A \code{function} will be called with a single argument,
the plot data. The return value must be a \code{data.frame}, and
will be used as the layer data. A \code{function} can be created
from a \code{formula} (e.g. \code{\textasciitilde{} head(.x, 10)}).

\item[\code{stat}] The statistical transformation to use on the data for this
layer, as a string.

\item[\code{position}] Position adjustment, either as a string, or the result of
a call to a position adjustment function.

\item[\code{priority}] Method used to perform point layout (see ggbeeswarm::position\_beeswarm)

\item[\code{cex}] Scaling for adjusting point spacing (see ggbeeswarm::position\_beeswarm)

\item[\code{groupOnX}] Should jitter be added to the x axis if TRUE or y axis if FALSE (the default NULL causes the function to guess which axis is the categorical one based on the number of unique entries in each) Refer to see ggbeeswarm::position\_beeswarm

\item[\code{dodge.width}] Amount by which points from different aesthetic groups will be dodged. This requires that one of the aesthetics is a factor. (see ggbeeswarm::position\_beeswarm)

\item[\code{...}] Other arguments passed on to \code{\LinkA{layer()}{layer()}}. These are
often aesthetics, used to set an aesthetic to a fixed value, like
\code{colour = "red"} or \code{size = 3}. They may also be parameters
to the paired geom/stat.

\item[\code{na.rm}] If \code{FALSE}, the default, missing values are removed with
a warning. If \code{TRUE}, missing values are silently removed.

\item[\code{show.legend}] logical. Should this layer be included in the legends?
\code{NA}, the default, includes if any aesthetics are mapped.
\code{FALSE} never includes, and \code{TRUE} always includes.
It can also be a named logical vector to finely select the aesthetics to
display.

\item[\code{inherit.aes}] If \code{FALSE}, overrides the default aesthetics,
rather than combining with them. This is most useful for helper functions
that define both data and aesthetics and shouldn't inherit behaviour from
the default plot specification, e.g. \code{\LinkA{borders()}{borders()}}.

\item[\code{raster.width}] Width of the result image (in inches). Default: deterined by the current device parameters.

\item[\code{raster.height}] Height of the result image (in inches). Default: deterined by the current device parameters.

\item[\code{raster.dpi}] Resolution of the result image.
\end{ldescription}
\end{Arguments}
%
\begin{Value}
geom\_beeswarm plot with rasterized layer
\end{Value}
%
\begin{Examples}
\begin{ExampleCode}
library(ggplot2)
library(ggrastr)

ggplot(mtcars) + geom_beeswarm_rast(aes(x = factor(cyl), y = mpg), raster.dpi = 600, cex = 1.5)

\end{ExampleCode}
\end{Examples}
\inputencoding{utf8}
\HeaderA{geom\_boxplot\_jitter}{This geom is similar to \code{\LinkA{geom\_boxplot}{geom.Rul.boxplot}}, but allows to jitter outlier points and to raster points layer.}{geom.Rul.boxplot.Rul.jitter}
%
\begin{Description}\relax
This geom is similar to \code{\LinkA{geom\_boxplot}{geom.Rul.boxplot}}, but allows to jitter outlier points and to raster points layer.
\end{Description}
%
\begin{Usage}
\begin{verbatim}
geom_boxplot_jitter(
  mapping = NULL,
  data = NULL,
  stat = "boxplot",
  position = "dodge",
  na.rm = FALSE,
  show.legend = NA,
  inherit.aes = TRUE,
  ...,
  outlier.jitter.width = NULL,
  outlier.jitter.height = 0,
  raster = FALSE,
  raster.dpi = 300,
  raster.width = NULL,
  raster.height = NULL
)
\end{verbatim}
\end{Usage}
%
\begin{Arguments}
\begin{ldescription}
\item[\code{mapping}] Set of aesthetic mappings created by \code{\LinkA{aes()}{aes()}} or
\code{\LinkA{aes\_()}{aes.Rul.()}}. If specified and \code{inherit.aes = TRUE} (the
default), it is combined with the default mapping at the top level of the
plot. You must supply \code{mapping} if there is no plot mapping.

\item[\code{data}] The data to be displayed in this layer. There are three
options:

If \code{NULL}, the default, the data is inherited from the plot
data as specified in the call to \code{\LinkA{ggplot()}{ggplot()}}.

A \code{data.frame}, or other object, will override the plot
data. All objects will be fortified to produce a data frame. See
\code{\LinkA{fortify()}{fortify()}} for which variables will be created.

A \code{function} will be called with a single argument,
the plot data. The return value must be a \code{data.frame}, and
will be used as the layer data. A \code{function} can be created
from a \code{formula} (e.g. \code{\textasciitilde{} head(.x, 10)}).

\item[\code{stat}] Use to override the default connection between
\code{geom\_boxplot} and \code{stat\_boxplot}.

\item[\code{position}] Position adjustment, either as a string, or the result of
a call to a position adjustment function.

\item[\code{na.rm}] If \code{FALSE}, the default, missing values are removed with
a warning. If \code{TRUE}, missing values are silently removed.

\item[\code{show.legend}] logical. Should this layer be included in the legends?
\code{NA}, the default, includes if any aesthetics are mapped.
\code{FALSE} never includes, and \code{TRUE} always includes.
It can also be a named logical vector to finely select the aesthetics to
display.

\item[\code{inherit.aes}] If \code{FALSE}, overrides the default aesthetics,
rather than combining with them. This is most useful for helper functions
that define both data and aesthetics and shouldn't inherit behaviour from
the default plot specification, e.g. \code{\LinkA{borders()}{borders()}}.

\item[\code{...}] Other arguments passed on to \code{\LinkA{layer()}{layer()}}. These are
often aesthetics, used to set an aesthetic to a fixed value, like
\code{colour = "red"} or \code{size = 3}. They may also be parameters
to the paired geom/stat.

\item[\code{outlier.jitter.width}] Amount of horizontal jitter. The jitter is added in both positive and negative directions,
so the total spread is twice the value specified here. Default: boxplot width.

\item[\code{outlier.jitter.height}] Amount of horizontal jitter. The jitter is added in both positive and negative directions,
so the total spread is twice the value specified here. Default: 0.

\item[\code{raster}] Should outlier points be rastered?.

\item[\code{raster.dpi}] Resolution of the rastered image. Ignored if \code{raster == FALSE}.

\item[\code{raster.width}] Width of the result image (in inches). Default: deterined by the current device parameters. Ignored if \code{raster == FALSE}.

\item[\code{raster.height}] Height of the result image (in inches). Default: deterined by the current device parameters. Ignored if \code{raster == FALSE}.
\end{ldescription}
\end{Arguments}
%
\begin{Value}
geom\_boxplot plot with rasterized layer
\end{Value}
%
\begin{Section}{Aesthetics}


\code{geom\_boxplot()} understands the following aesthetics (required aesthetics are in bold):
\begin{itemize}

\item{} \strong{\code{x} \emph{or} \code{y}}
\item{} \strong{\code{lower} \emph{or} \code{xlower}}
\item{} \strong{\code{upper} \emph{or} \code{xupper}}
\item{} \strong{\code{middle} \emph{or} \code{xmiddle}}
\item{} \strong{\code{ymin} \emph{or} \code{xmin}}
\item{} \strong{\code{ymax} \emph{or} \code{xmax}}
\item{} \code{alpha}
\item{} \code{colour}
\item{} \code{fill}
\item{} \code{group}
\item{} \code{linetype}
\item{} \code{shape}
\item{} \code{size}
\item{} \code{weight}

\end{itemize}

Learn more about setting these aesthetics in \code{vignette("ggplot2-specs")}.

\end{Section}
%
\begin{Examples}
\begin{ExampleCode}
library(ggplot2)
library(ggrastr)

yvalues = rt(1000, df=3)
xvalues = as.factor(1:1000 %% 2)
ggplot() + geom_boxplot_jitter(aes(y=yvalues, x=xvalues), outlier.jitter.width = 0.1, raster = TRUE)

\end{ExampleCode}
\end{Examples}
\inputencoding{utf8}
\HeaderA{geom\_jitter\_rast}{This geom is similar to \code{\LinkA{geom\_jitter}{geom.Rul.jitter}}, but creates a raster layer}{geom.Rul.jitter.Rul.rast}
%
\begin{Description}\relax
This geom is similar to \code{\LinkA{geom\_jitter}{geom.Rul.jitter}}, but creates a raster layer
\end{Description}
%
\begin{Usage}
\begin{verbatim}
geom_jitter_rast(
  mapping = NULL,
  data = NULL,
  stat = "identity",
  position = "jitter",
  width = NULL,
  height = NULL,
  seed = NA,
  ...,
  na.rm = FALSE,
  show.legend = NA,
  inherit.aes = TRUE,
  raster.width = NULL,
  raster.height = NULL,
  raster.dpi = 300
)
\end{verbatim}
\end{Usage}
%
\begin{Arguments}
\begin{ldescription}
\item[\code{mapping}] Set of aesthetic mappings created by \code{\LinkA{aes()}{aes()}} or
\code{\LinkA{aes\_()}{aes.Rul.()}}. If specified and \code{inherit.aes = TRUE} (the
default), it is combined with the default mapping at the top level of the
plot. You must supply \code{mapping} if there is no plot mapping.

\item[\code{data}] The data to be displayed in this layer. There are three
options:

If \code{NULL}, the default, the data is inherited from the plot
data as specified in the call to \code{\LinkA{ggplot()}{ggplot()}}.

A \code{data.frame}, or other object, will override the plot
data. All objects will be fortified to produce a data frame. See
\code{\LinkA{fortify()}{fortify()}} for which variables will be created.

A \code{function} will be called with a single argument,
the plot data. The return value must be a \code{data.frame}, and
will be used as the layer data. A \code{function} can be created
from a \code{formula} (e.g. \code{\textasciitilde{} head(.x, 10)}).

\item[\code{stat}] The statistical transformation to use on the data for this
layer, as a string.

\item[\code{position}] Position adjustment, either as a string, or the result of
a call to a position adjustment function.

\item[\code{width}] Amount of vertical and horizontal jitter. The jitter is added in both positive and negative directions, so the total spread is twice the value specified here. Refer to ggplot2::position\_jitter.

\item[\code{height}] Amount of vertical and horizontal jitter. The jitter is added in both positive and negative directions, so the total spread is twice the value specified here. Refer to ggplot2::position\_jitter.

\item[\code{seed}] A random seed to make the jitter reproducible. Refer to ggplot2::position\_jitter.

\item[\code{...}] Other arguments passed on to \code{\LinkA{layer()}{layer()}}. These are
often aesthetics, used to set an aesthetic to a fixed value, like
\code{colour = "red"} or \code{size = 3}. They may also be parameters
to the paired geom/stat.

\item[\code{na.rm}] If \code{FALSE}, the default, missing values are removed with
a warning. If \code{TRUE}, missing values are silently removed.

\item[\code{show.legend}] logical. Should this layer be included in the legends?
\code{NA}, the default, includes if any aesthetics are mapped.
\code{FALSE} never includes, and \code{TRUE} always includes.
It can also be a named logical vector to finely select the aesthetics to
display.

\item[\code{inherit.aes}] If \code{FALSE}, overrides the default aesthetics,
rather than combining with them. This is most useful for helper functions
that define both data and aesthetics and shouldn't inherit behaviour from
the default plot specification, e.g. \code{\LinkA{borders()}{borders()}}.

\item[\code{raster.width}] Width of the result image (in inches). Default: deterined by the current device parameters.

\item[\code{raster.height}] Height of the result image (in inches). Default: deterined by the current device parameters.

\item[\code{raster.dpi}] Resolution of the result image.
\end{ldescription}
\end{Arguments}
%
\begin{Value}
geom\_point\_rast plot with rasterized layer
\end{Value}
%
\begin{Section}{Aesthetics}


\code{geom\_point()} understands the following aesthetics (required aesthetics are in bold):
\begin{itemize}

\item{} \strong{\code{x}}
\item{} \strong{\code{y}}
\item{} \code{alpha}
\item{} \code{colour}
\item{} \code{fill}
\item{} \code{group}
\item{} \code{shape}
\item{} \code{size}
\item{} \code{stroke}

\end{itemize}

Learn more about setting these aesthetics in \code{vignette("ggplot2-specs")}.

\end{Section}
%
\begin{Examples}
\begin{ExampleCode}
library(ggplot2)
library(ggrastr)

ggplot(mpg) + geom_jitter_rast(aes(x = factor(cyl), y = hwy), raster.dpi = 600)

\end{ExampleCode}
\end{Examples}
\inputencoding{utf8}
\HeaderA{geom\_point\_rast}{This geom is similar to \code{\LinkA{geom\_point}{geom.Rul.point}}, but creates a raster layer}{geom.Rul.point.Rul.rast}
%
\begin{Description}\relax
This geom is similar to \code{\LinkA{geom\_point}{geom.Rul.point}}, but creates a raster layer
\end{Description}
%
\begin{Usage}
\begin{verbatim}
geom_point_rast(
  mapping = NULL,
  data = NULL,
  stat = "identity",
  position = "identity",
  ...,
  na.rm = FALSE,
  show.legend = NA,
  inherit.aes = TRUE,
  raster.width = NULL,
  raster.height = NULL,
  raster.dpi = 300
)
\end{verbatim}
\end{Usage}
%
\begin{Arguments}
\begin{ldescription}
\item[\code{mapping}] Set of aesthetic mappings created by \code{\LinkA{aes()}{aes()}} or
\code{\LinkA{aes\_()}{aes.Rul.()}}. If specified and \code{inherit.aes = TRUE} (the
default), it is combined with the default mapping at the top level of the
plot. You must supply \code{mapping} if there is no plot mapping.

\item[\code{data}] The data to be displayed in this layer. There are three
options:

If \code{NULL}, the default, the data is inherited from the plot
data as specified in the call to \code{\LinkA{ggplot()}{ggplot()}}.

A \code{data.frame}, or other object, will override the plot
data. All objects will be fortified to produce a data frame. See
\code{\LinkA{fortify()}{fortify()}} for which variables will be created.

A \code{function} will be called with a single argument,
the plot data. The return value must be a \code{data.frame}, and
will be used as the layer data. A \code{function} can be created
from a \code{formula} (e.g. \code{\textasciitilde{} head(.x, 10)}).

\item[\code{stat}] The statistical transformation to use on the data for this
layer, as a string.

\item[\code{position}] Position adjustment, either as a string, or the result of
a call to a position adjustment function.

\item[\code{...}] Other arguments passed on to \code{\LinkA{layer()}{layer()}}. These are
often aesthetics, used to set an aesthetic to a fixed value, like
\code{colour = "red"} or \code{size = 3}. They may also be parameters
to the paired geom/stat.

\item[\code{na.rm}] If \code{FALSE}, the default, missing values are removed with
a warning. If \code{TRUE}, missing values are silently removed.

\item[\code{show.legend}] logical. Should this layer be included in the legends?
\code{NA}, the default, includes if any aesthetics are mapped.
\code{FALSE} never includes, and \code{TRUE} always includes.
It can also be a named logical vector to finely select the aesthetics to
display.

\item[\code{inherit.aes}] If \code{FALSE}, overrides the default aesthetics,
rather than combining with them. This is most useful for helper functions
that define both data and aesthetics and shouldn't inherit behaviour from
the default plot specification, e.g. \code{\LinkA{borders()}{borders()}}.

\item[\code{raster.width}] Width of the result image (in inches). Default: deterined by the current device parameters.

\item[\code{raster.height}] Height of the result image (in inches). Default: deterined by the current device parameters.

\item[\code{raster.dpi}] Resolution of the result image.
\end{ldescription}
\end{Arguments}
%
\begin{Value}
geom\_point plot with rasterized layer
\end{Value}
%
\begin{Section}{Aesthetics}


\code{geom\_point()} understands the following aesthetics (required aesthetics are in bold):
\begin{itemize}

\item{} \strong{\code{x}}
\item{} \strong{\code{y}}
\item{} \code{alpha}
\item{} \code{colour}
\item{} \code{fill}
\item{} \code{group}
\item{} \code{shape}
\item{} \code{size}
\item{} \code{stroke}

\end{itemize}

Learn more about setting these aesthetics in \code{vignette("ggplot2-specs")}.

\end{Section}
%
\begin{Examples}
\begin{ExampleCode}
library(ggplot2)
library(ggrastr)

ggplot() + geom_point_rast(aes(x=rnorm(1000), y=rnorm(1000)), raster.dpi=600)

\end{ExampleCode}
\end{Examples}
\inputencoding{utf8}
\HeaderA{geom\_quasirandom\_rast}{This geom is similar to \code{\LinkA{geom\_quasirandom}{geom.Rul.quasirandom}}, but creates a raster layer}{geom.Rul.quasirandom.Rul.rast}
%
\begin{Description}\relax
This geom is similar to \code{\LinkA{geom\_quasirandom}{geom.Rul.quasirandom}}, but creates a raster layer
\end{Description}
%
\begin{Usage}
\begin{verbatim}
geom_quasirandom_rast(
  mapping = NULL,
  data = NULL,
  stat = "identity",
  position = "quasirandom",
  width = NULL,
  varwidth = FALSE,
  bandwidth = 0.5,
  nbins = NULL,
  method = "quasirandom",
  groupOnX = NULL,
  dodge.width = 0,
  ...,
  na.rm = FALSE,
  show.legend = NA,
  inherit.aes = TRUE,
  raster.width = NULL,
  raster.height = NULL,
  raster.dpi = 300
)
\end{verbatim}
\end{Usage}
%
\begin{Arguments}
\begin{ldescription}
\item[\code{mapping}] Set of aesthetic mappings created by \code{\LinkA{aes()}{aes()}} or
\code{\LinkA{aes\_()}{aes.Rul.()}}. If specified and \code{inherit.aes = TRUE} (the
default), it is combined with the default mapping at the top level of the
plot. You must supply \code{mapping} if there is no plot mapping.

\item[\code{data}] The data to be displayed in this layer. There are three
options:

If \code{NULL}, the default, the data is inherited from the plot
data as specified in the call to \code{\LinkA{ggplot()}{ggplot()}}.

A \code{data.frame}, or other object, will override the plot
data. All objects will be fortified to produce a data frame. See
\code{\LinkA{fortify()}{fortify()}} for which variables will be created.

A \code{function} will be called with a single argument,
the plot data. The return value must be a \code{data.frame}, and
will be used as the layer data. A \code{function} can be created
from a \code{formula} (e.g. \code{\textasciitilde{} head(.x, 10)}).

\item[\code{stat}] The statistical transformation to use on the data for this
layer, as a string.

\item[\code{position}] Position adjustment, either as a string, or the result of
a call to a position adjustment function.

\item[\code{width}] the maximum amount of spread (default: 0.4)

\item[\code{varwidth}] vary the width by the relative size of each group

\item[\code{bandwidth}] the bandwidth adjustment to use when calculating density
Smaller numbers (< 1) produce a tighter "fit". (default: 0.5)

\item[\code{nbins}] the number of bins used when calculating density (has little effect with quasirandom/random distribution)

\item[\code{method}] the method used for distributing points (quasirandom, pseudorandom, smiley or frowney)

\item[\code{groupOnX}] if TRUE then jitter is added to the x axis and if FALSE jitter is added to the y axis. Prior to v0.6.0, the default NULL causes the function to guess which axis is the categorical one based on the number of unique entries in each. This could result in unexpected results when the x variable has few unique values and so in v0.6.0 the default was changed to always jitter on the x axis unless groupOnX=FALSE. Also consider \code{\LinkA{coord\_flip}{coord.Rul.flip}}.

\item[\code{dodge.width}] Amount by which points from different aesthetic groups will be dodged. This requires that one of the aesthetics is a factor.

\item[\code{...}] Other arguments passed on to \code{\LinkA{layer()}{layer()}}. These are
often aesthetics, used to set an aesthetic to a fixed value, like
\code{colour = "red"} or \code{size = 3}. They may also be parameters
to the paired geom/stat.

\item[\code{na.rm}] If \code{FALSE}, the default, missing values are removed with
a warning. If \code{TRUE}, missing values are silently removed.

\item[\code{show.legend}] logical. Should this layer be included in the legends?
\code{NA}, the default, includes if any aesthetics are mapped.
\code{FALSE} never includes, and \code{TRUE} always includes.
It can also be a named logical vector to finely select the aesthetics to
display.

\item[\code{inherit.aes}] If \code{FALSE}, overrides the default aesthetics,
rather than combining with them. This is most useful for helper functions
that define both data and aesthetics and shouldn't inherit behaviour from
the default plot specification, e.g. \code{\LinkA{borders()}{borders()}}.

\item[\code{raster.width}] Width of the result image (in inches). Default: deterined by the current device parameters.

\item[\code{raster.height}] Height of the result image (in inches). Default: deterined by the current device parameters.

\item[\code{raster.dpi}] Resolution of the result image.
\end{ldescription}
\end{Arguments}
%
\begin{Value}
geom\_quasirandom plot with rasterized layer
\end{Value}
%
\begin{Section}{Aesthetics}


\code{geom\_point()} understands the following aesthetics (required aesthetics are in bold):
\begin{itemize}

\item{} \strong{\code{x}}
\item{} \strong{\code{y}}
\item{} \code{alpha}
\item{} \code{colour}
\item{} \code{fill}
\item{} \code{group}
\item{} \code{shape}
\item{} \code{size}
\item{} \code{stroke}

\end{itemize}

Learn more about setting these aesthetics in \code{vignette("ggplot2-specs")}.

\end{Section}
%
\begin{Examples}
\begin{ExampleCode}
library(ggplot2)
library(ggrastr)

ggplot(mtcars) + geom_quasirandom_rast(aes(x = factor(cyl), y = mpg), raster.dpi = 600)

\end{ExampleCode}
\end{Examples}
\inputencoding{utf8}
\HeaderA{geom\_tile\_rast}{This geom is similar to \code{\LinkA{geom\_tile}{geom.Rul.tile}}, but creates a raster layer}{geom.Rul.tile.Rul.rast}
%
\begin{Description}\relax
This geom is similar to \code{\LinkA{geom\_tile}{geom.Rul.tile}}, but creates a raster layer
\end{Description}
%
\begin{Usage}
\begin{verbatim}
geom_tile_rast(
  mapping = NULL,
  data = NULL,
  stat = "identity",
  position = "identity",
  ...,
  na.rm = FALSE,
  show.legend = NA,
  inherit.aes = TRUE,
  raster.width = NULL,
  raster.height = NULL,
  raster.dpi = 300
)
\end{verbatim}
\end{Usage}
%
\begin{Arguments}
\begin{ldescription}
\item[\code{mapping}] Set of aesthetic mappings created by \code{\LinkA{aes()}{aes()}} or
\code{\LinkA{aes\_()}{aes.Rul.()}}. If specified and \code{inherit.aes = TRUE} (the
default), it is combined with the default mapping at the top level of the
plot. You must supply \code{mapping} if there is no plot mapping.

\item[\code{data}] The data to be displayed in this layer. There are three
options:

If \code{NULL}, the default, the data is inherited from the plot
data as specified in the call to \code{\LinkA{ggplot()}{ggplot()}}.

A \code{data.frame}, or other object, will override the plot
data. All objects will be fortified to produce a data frame. See
\code{\LinkA{fortify()}{fortify()}} for which variables will be created.

A \code{function} will be called with a single argument,
the plot data. The return value must be a \code{data.frame}, and
will be used as the layer data. A \code{function} can be created
from a \code{formula} (e.g. \code{\textasciitilde{} head(.x, 10)}).

\item[\code{stat}] The statistical transformation to use on the data for this
layer, as a string.

\item[\code{position}] Position adjustment, either as a string, or the result of
a call to a position adjustment function.

\item[\code{...}] Other arguments passed on to \code{\LinkA{layer()}{layer()}}. These are
often aesthetics, used to set an aesthetic to a fixed value, like
\code{colour = "red"} or \code{size = 3}. They may also be parameters
to the paired geom/stat.

\item[\code{na.rm}] If \code{FALSE}, the default, missing values are removed with
a warning. If \code{TRUE}, missing values are silently removed.

\item[\code{show.legend}] logical. Should this layer be included in the legends?
\code{NA}, the default, includes if any aesthetics are mapped.
\code{FALSE} never includes, and \code{TRUE} always includes.
It can also be a named logical vector to finely select the aesthetics to
display.

\item[\code{inherit.aes}] If \code{FALSE}, overrides the default aesthetics,
rather than combining with them. This is most useful for helper functions
that define both data and aesthetics and shouldn't inherit behaviour from
the default plot specification, e.g. \code{\LinkA{borders()}{borders()}}.

\item[\code{raster.width}] Width of the result image (in inches). Default: deterined by the current device parameters.

\item[\code{raster.height}] Height of the result image (in inches). Default: deterined by the current device parameters.

\item[\code{raster.dpi}] Resolution of the result image.
\end{ldescription}
\end{Arguments}
%
\begin{Value}
geom\_tile plot with rasterized layer
\end{Value}
%
\begin{Section}{Aesthetics}


\code{geom\_tile()} understands the following aesthetics (required aesthetics are in bold):
\begin{itemize}

\item{} \strong{\code{x}}
\item{} \strong{\code{y}}
\item{} \code{alpha}
\item{} \code{colour}
\item{} \code{fill}
\item{} \code{group}
\item{} \code{height}
\item{} \code{linetype}
\item{} \code{size}
\item{} \code{width}

\end{itemize}

Learn more about setting these aesthetics in \code{vignette("ggplot2-specs")}.

\end{Section}
%
\begin{Examples}
\begin{ExampleCode}
library(ggplot2)
library(ggrastr)

coords <- expand.grid(1:100, 1:100)
coords$Value <- 1 / apply(as.matrix(coords), 1, function(x) sum((x - c(50, 50))^2)^0.01)
ggplot(coords) + geom_tile_rast(aes(x=Var1, y=Var2, fill=Value))

\end{ExampleCode}
\end{Examples}
\inputencoding{utf8}
\HeaderA{theme\_pdf}{Pretty theme}{theme.Rul.pdf}
%
\begin{Description}\relax
Pretty theme
\end{Description}
%
\begin{Usage}
\begin{verbatim}
theme_pdf(show.ticks = TRUE, legend.pos = NULL)
\end{verbatim}
\end{Usage}
%
\begin{Arguments}
\begin{ldescription}
\item[\code{show.ticks}] Show x- and y-ticks.

\item[\code{legend.pos}] Vector with x and y position of the legend.
\end{ldescription}
\end{Arguments}
%
\begin{Value}
ggplot2 with plot ticks and positioned legend
\end{Value}
%
\begin{Examples}
\begin{ExampleCode}
library(ggplot2)
library(ggrastr)

data = rnorm(100)
colors = (1:100/100)
ggplot() + geom_point(aes(x=data, y=data, color=colors)) + theme_pdf(FALSE, legend.pos=c(1, 1))

\end{ExampleCode}
\end{Examples}
\printindex{}
\end{document}
